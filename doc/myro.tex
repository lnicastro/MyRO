% ----------------------------------------------------------------------^
% Copyright (C) 2004, 2005, 2006, 2007 Giorgio Calderone <gcalderone@ifc.inaf.it>
% 
% This file is part of MYRO.
% 
% MYRO is free software; you can redistribute it and/or modify
% it under the terms of the GNU General Public License as published by
% the Free Software Foundation; either version 2 of the License, or
% (at your option) any later version.
% 
% MYRO is distributed in the hope that it will be useful,
% but WITHOUT ANY WARRANTY; without even the implied warranty of
% MERCHANTABILITY or FITNESS FOR A PARTICULAR PURPOSE.  See the
% GNU General Public License for more details.
% 
% You should have received a copy of the GNU General Public License
% along with MYRO; if not, write to the Free Software
% Foundation, Inc., 51 Franklin St, Fifth Floor, Boston, MA  02110-1301  USA
% 
% ----------------------------------------------------------------------$


\documentclass[12pt,titlepage]{article}

\usepackage[english]{babel}
\usepackage{graphicx}
\usepackage{framed}
\usepackage{listings}

\usepackage{a4wide}
\usepackage{fancyhdr}

\usepackage[usenames,dvipsnames]{xcolor} % for hyperref
\usepackage{hyperref}
\hypersetup{colorlinks,urlcolor=MidnightBlue,linkcolor=Brown}

\pagestyle{fancy}
\renewcommand{\footrulewidth}{0.4pt}

%LN
%\renewcommand{\sectionmark}[1]{\markright{\thesection.\ #1}}
%\renewcommand{\sectionmark}[1]{\markright{#1}}
%\renewcommand{\subsectionmark}[1]{\markright{#1}}
%\parindent 0pt


%\textheight 22cm
%\textwidth 16.5cm
%\topmargin 0cm
%\oddsidemargin -0.5cm
%\evensidemargin -0.5cm
%\sloppy

\newcommand{\myro}{\textbf{MyRO} }
\newcommand{\myrO}{\textbf{MyRO}}

\def\ver{Ver. \input{../version}}
\def\git{ \url{https://github.com/lnicastro/MyRO} }

\newcommand{\syntax}[1]
{
  \bigskip
  \noindent
  \textbf{Syntax: } \\ 
  \indent \texttt{#1}
}

\newenvironment{parameters}
{
  \bigskip
  \noindent
  \textbf{Parameters:}
  \begin{enumerate}
}
{
  \end{enumerate}
}

\newcommand{\param}[2]
{
  \item \textit{#1} \texttt{#2} 
}

\newcommand{\return}[1]
{
  \bigskip
  \noindent
  \textbf{Return value} (\texttt{#1}): \\
  \indent
}




\begin{document}
\lstloadlanguages{[ISO]C++}
\lstset{
  language=c++,
  frame=leftline,
  escapeinside={//(}{)},
  morekeywords={string},
  keywordstyle=\bfseries, stringstyle=\ttfamily, commentstyle=\textit,
  numbers=left, numberstyle=\tiny, stepnumber=1, numbersep=5pt,
  showstringspaces=false, emphstyle=\underbar,
  captionpos=b,
  aboveskip=3mm,
  belowskip=3mm,
  xleftmargin=0.5cm,
  emph={
        TINY,SMALL,MEDIUM,INT,BIGINT,FLOAT,DOUBLE,STRING,TIME,TINY_BLOB,BLOB
        ,AccumBuffer
        ,B64_Codec
        ,BaseThread
        ,Binary
        ,Client
        ,ClientInfo
        ,Column
        ,CommandParser
        ,Conf
        ,Coosys
        ,Data
        ,DBConn
        ,Definitions
        ,Description
        ,Dynamic_Array
        ,Element
        ,Env
        ,Event
        ,Field
        ,FieldRef
        ,Fits
        ,Group
        ,HostInfo
        ,Info
        ,Link
        ,LocalThread
        ,Max
        ,Min
        ,NetInterface
        ,NodePointer
        ,Option
        ,Param
        ,ParamRef
        ,Parser_Stream
        ,Parser_Table
        ,Parser_Tree
        ,Pipe
        ,Query
        ,Record
        ,RecordSet
        ,Resource
        ,Row
        ,Serializable
        ,Server
        ,ServerSocket
        ,Socket
        ,Stream
        ,Synchro
        ,Table
        ,Tabledata
        ,Thread
        ,URLReader
        ,UserThread
        ,Values
        ,VOTable
        ,VOTableReaderSplit
        ,Writer_Stream
        ,mcs
        ,mcsStart
        ,mcsWait
        ,mcsCustomStart
        }
}


\title{
%
\textbf{MYRO \\ My Record Oriented \\ privilege system}
%
\centerline{\rule{\textwidth}{0.4pt}}
%
}
\bigskip
\author{
  \sc{Giorgio Calderone} \\
  %\footnote{Giorgio Calderone $<$giorgio.calderone@inaf.it$>$}  \\
  \sc{INAF - OA Trieste, Italy} \\ \\
  \sc{Luciano Nicastro} \\
  %\footnote{Luciano Nicastro $<$luciano.nicastro@inaf.it$>$} \\
  \sc{INAF - OAS Bologna, Italy} \\
 }
\date{\today \\
      \ver \\
      \git \\
      \centerline{\rule{\textwidth}{0.4pt}}
     }
\maketitle

\thispagestyle{empty}
~

\vfill
{\it\small This page intentionally left blank.}

\newpage
\pagenumbering{roman}

\tableofcontents

\newpage
\pagenumbering{arabic}
%\mainmatter

\section{Introduction}

\section{\myrO: My Record Oriented privilege system}
\myro is the name we use to refer to a technique used to implement a
database privilege system on a per-record basis on the MySQL database.
Actually all database servers implement a privilege system based on
tables or columns, that is if a user has grants to access a certain
table (or a table's column) he can access all records of that table
(or that table's column). \myro lets you specify grants on a record
level so a user can access only those records it is allowed to
``select/update/delete''. A
consequence of this is that different users reading the same table
will see different records. The grant mechanism provided by \myro is
similar to that of a Unix file system, that is each record belongs to
a ``owner'' and a ``group'' and has three sets of permissions
associated (for the owner, the users belonging to the group and all
other users) that specify if that record can be read and/or written.

\smallskip
The software components of \myro are a Perl script used to perform
administrative tasks, a C library and a set of MySQL functions. The
process of protecting tables is completely transparent to the final
user, that is once \myro has been installed and configured by the
database administrator, users can access the database without even
know that \myro is working.
%
%
%MySQL\footnote{http://www.mysql.com} provide a sophisticated privilege
%system able to give users grant on a per database, table and column
%basis. Nevertheless it doesn’t provide any feature to implement a per-
%record privilege. In this document we’ll explain how to reach this
%goal using \myro, a record oriented privilege system. The goal of
%\myro is to implement a privilege system similar to the one used in
%any Unix file system. With each record of a protected table will be
%stored an extra information regarding the user and group who owns that
%record, and the permission specification. The last information is a
%number which specifies the \verb|READ| or \verb|WRITE| permission for
%the owner of the record, users of the group to which the record
%belongs, and all other users.  A typical permission specification
%looks like this:
%%
%\begin{verbatim}
%    rwr---
%\end{verbatim}
%%
%This sequence of six character represent the permission specification
%for the owner itself (first two character), the group (third and
%fourth), and all other users (fifth and sixth). The letter ’r’ means
%read permission, the letter ’w’ means write permission.



%---------------------------------------------------------------------
\newpage
\section{\myro installation}

The \myro software library is distributed in a \verb|tar.gz| package
and via GitHub.
You can find the latest version at \git. To unpack the package simply
issue the command:
%
\begin{verbatim}
    tar xvzf myro-x.y.z.tar.gz
\end{verbatim}
%
where \verb|x|, \verb|y|, \verb|z| are the version number (namely the
first number is the major revision, the second number is the version,
and third number is the subversion). A directory named
\verb|myro-x.y.z| will be created containing all sources code as well
as the documentation and the scripts needed to install \myrO. Before
installing \myro you should check that all mandatory dependencies are
satisfied (see Sect. \ref{sec:Dependencies}), then you must follow a
three step procedure: Configure, Compile and Installing \myro.

\subsection{Dependencies}
\label{sec:Dependencies}
The only mandatory packages required by \myro are:

\begin{itemize}
\item MySQL sources (\verb|http://www.mysql.org|, version 5.1.20 or later);
\item Perl (\verb|http://www.perl.com/|);
\item The \verb|DBD::mysql| perl module;
\end{itemize}

\noindent
The \verb|DBD::mysql| perl module can be easily installed through the
\verb|cpan| utility issuing the following command:
%
\begin{verbatim}
  install DBD::mysql
\end{verbatim}
%
See the \verb|cpan| documentation for further informations. If these
package are not already installed in the system you should install
them before continuing.


\subsection{Installing \myro}
\label{sec:installing}

\subsubsection{Configure}
\label{sec:configuremyro}
Configuring \myro means checking your system for compatibilities,
search for include files and libraries, and finally produce all
necessary \textsf{Makefile}s needed to compile \myrO. This is done
automatically by the distributed \verb|configure| script. Typically
you can use this script without any option, as follows:

\begin{verbatim}
    ./configure
\end{verbatim}

\noindent
Anyway \verb|configure| has a lot of options and switches (type
\verb|configure --help| for a list) to customize the compilation step.
For further documentation see the \verb|INSTALL| file.

%
\subsubsection{Compile}
\label{sec:compilemyro}
To compile \myrO, once the \verb|configure| script has been correctly
executed, simply issue the command:
%
\begin{verbatim}
    make
\end{verbatim}
%
If you got errors while compiling check Sect. \ref{sec:configuremyro}
and the \verb|INSTALL| file.


\subsubsection{Install}
\label{sec:installmyro}
If \myro has been correctly compiled you can install with the command:
%
\begin{verbatim}
    make install
\end{verbatim}
%
If your account doesn't have the permission to write in the path where
it should be installed then you'll get an error. In this case you
should login as ``root'' and retry.


%---------------------------------------------------------------------
\newpage
\section{How it works}
Suppose the table to be protected is called \verb|mytable_data|. \myro
adds three fields to the table to store information about the owner
and group to which the records belong, and for the permissions. Then
it installs three triggers on the table associated with the INSERT,
UPDATE and DELETE events, and finally creates a view\footnote{Views
  are database objects just like tables, except that they don't
  require disk space because their data is read from the actual tables.}
with a custom name (for example \verb|mytable|) with a simple \verb|SELECT *|
statement and a \verb|WHERE| clause that filter the records readable by the
user executing the query. This way the next time users will access
\verb|mytable| they will access the view, not the underlying table,
and they will see only the records they're allowed to read or write.
When a user tries to write on a table the corresponding trigger will
be activated for each record being modified, to check if the user has
grants to modify the record.

%
\subsection{Users, groups and permissions}
Users and groups used in \myro are completely analogous to those of a
Unix file system. Each MySQL account has a unique user id (\verb|uid|)
and can be a member of any number of groups, each associated with a
unique group id (\verb|gid|). Furthermore to each MySQL users is
associated a flag that tells if the user is a ``super-user'', in this
case all permission checking will be disabled. \myro automatically
handles its internal tables where all \verb|uid|, group definition
and group membership are stored. Also each record in a table protected
by \myro has three fields which contain the \verb|uid| of the owner,
the \verb|gid| of the group owning that record, and a permission
specification, that is a numerical code that specify if that record
can be read and/or written by three categories of users: owner of the
record, member of the group to which the record belongs to and all
other users. \myro uses all these information to determine if a
record can be read and/or written. The permission specification is a
sequence of 6 bits whose meaning are as follows (from the least
to the most significant bit):
%
\begin{itemize}
\item write permission for owner;
\item read permission for owner;
\item write permission for members of the group;
\item read permission for members of the group;
\item write permission for all other users;
\item read permission for all other users;
\end{itemize}

\myro has a function to properly convert a permission specification to
a more readable string representation of the grants:
%
\begin{verbatim}
mysql> select myro.fmtPerm(43);
+------------------+
| myro.fmtPerm(43) |
+------------------+
| rwr-r-           | 
+------------------+
1 row in set (0.01 sec)
\end{verbatim}
%
in which the two first characters refer to the owner, the third and
fourth to the members of the group and the last two characters to all
other users. Also \myro automatically creates a group named
\verb|anygroup|. Members of this group are automatically members of
any other group. This feature is necessary when you are dealing with
many groups creation and destroy, and you know that a user must be
member of all of this groups. Assigning the user to this group means
it is member of any group, even if they are created after the user has
been assigned to \verb|anygroup|.


%---------------------------------------------------------------------
\newpage
\section{\myro usage}
All administrative tasks related to \myrO, like protecting tables or
manipulating groups, can be performed using the the \verb|myro|
script. All available options can be displayed using the command:
%
\begin{verbatim}
  myro --help
\end{verbatim}
%
Before using \myro it is necessary to install the functions in the
MySQL database server (version $>= 5.1.20$) with the command:
%
\begin{verbatim}
  myro --install
\end{verbatim}
%
The password of the MySQL root account will be asked. This command
also creates a database named \verb|myro| which contains all \myro
internal tables.

%
\subsection{Implementing the privilege system}
Once \myro has been installed in the database server all MySQL users
will already have a \verb|uid| and a default group to which they
belong whose name is the same as the account username. Furthermore
the group \verb|anygroup| will automatically be created. The steps
needed to implement the privilege system are as follows:
%
\begin{itemize}
\item creates all needed groups;
\item modify users membership to groups;
\item protect tables;
\item check that all users' grants are consistent.
\end{itemize}
%
These operations are discussed in the next sections.

%
\subsubsection{Handle groups}
A new group can be created using the following command:
%
\begin{verbatim}
myro --addgroup <GroupName> [<Description>]
\end{verbatim}
%
To delete a group use the command:
%
\begin{verbatim}
myro --delgroup <GroupName>
\end{verbatim}
%
To see a list of defined groups issue the following SQL statement from
a MySQL terminal:
%
\begin{verbatim}
call myro.groups();
\end{verbatim}

%
\subsubsection{Modify user's properties}
With the following command it is possible to modify the default group
to which a user belongs, specify if it is a ``super-user'', provide a
description and an e-mail address:
%
\begin{verbatim}
myro --moduser <UserName> [<Group> [<Su> [<Description> [<Email>]]]]
\end{verbatim}
%
To assign a user to a group use the command:
%
\begin{verbatim}
myro --assign <UserName> <Group>
\end{verbatim}
%
To delete a user account you should drop the entire MySQL user
account. To see the list of users along with their properties issue
the following SQL statement from a MySQL terminal:
%
\begin{verbatim}
call myro.users();
\end{verbatim}

%
\subsubsection{Protect a table}
To protect a table with \myro issue the command:
%
\begin{verbatim}
myro --protect <DBName> <Table> <View>
\end{verbatim}
%
where the first two arguments are the database which contains the
table, and the table name to protect. The third argument is the name
of the view that will be used to access the table. This command will
add three fields in the table:
%
\begin{itemize}
\item \verb|my_uid|: to store the user ID of record owner;
\item \verb|my_gid|: to store the group ID to which the record belongs;
\item \verb|my_perm|: to store the record permissions.
\end{itemize}
%
These fields are all of type \verb|TINYINT UNSIGNED| so each record
will require 3 more bytes on disk. Actually this field type is fixed,
thus only 256 users and groups can be defined, but this feature may
change in future releases. If the table being protected by myro
already contains records the corresponding \verb|my_uid|,
\verb|my_gid| and \verb|my_perm| fields will contain NULL values, this
means that these records will be accessible only from ``super-users''.
To change the owner of these records open a MySQL terminal as user
``root'' and issue the following statement:
%
\begin{verbatim}
UPDATE <table> SET my_uid  = myro.uid(), 
                   my_gid  = myro.defgid(),
                   my_perm = myro.defperm();
\end{verbatim}
%
where \verb|<table>| is the name of the table being protected. For those
records that will be inserted after the table has been protected, this
fields will be automatically populated with the correct values. The
command \verb|myro --protect| will also create a view with the same
structure as the underlying table whose purpose is to filter records
that can be read (etc.) by a user. Users should now use this view to access
the data instead of the real table.

%
\subsubsection{Check users' grants}
\label{sec:myro_Check users' grants}
All grants relative to the protected table should be removed for any
users so that the only way to access the data is to use the view
created by \myrO. To check that all user's grants are compatible with
this requirement you can simply execute the \verb|myro| script without
any argument, so that if some user can directly access one of the
protected tables a warning will be given. On the other hand users
should have grants to access \myrO's views. To give users the correct
grants to access a \myro view issue the command:
%
\begin{verbatim}
myro --grant <UserName> <Host> <DBName> <View>
\end{verbatim}
%

\subsection{Extended views}
The view created by \myro has the same structure as the underlying
table without the \verb|my_uid|, \verb|my_gid| and \verb|my_perm|
fields, thus it is impossible to read the information stored in these
fields. Typically these information are not needed by the users, however
there may be some cases in which it is necessary to show these
information. For this purpose \myro can create three additional views
which shows these information in various formats. To create this
views issue the command:
%
\begin{verbatim}
myro --extended <DBName> <Table>
\end{verbatim}


%---------------------------------------------------------------------
\newpage
\section{Reference for functions and procedures}
Below there is a list of all functions and procedure created with
\myrO, with their usage and parameters. Here we'll use some common
abbreviations which resemble the ones used in Unix filesystems:

\begin{itemize}
\item \verb|uid|: an integer which represent the user ID;
\item \verb|gid|: an integer which represent the group ID;
\item \verb|defgid|: an integer which represent the default group ID
  of any newly inserted record for a given user;
\item \verb|defperm|: an integer which represent the default
  permission specification of any newly inserted record for a given
  user.
\end{itemize}



\subsection{myro.chkPerm}
Check if a user can access a record for a read or write operation. The
first parameter is the ID of the user who wants to access the
record, the next three parameters are the value of the \verb|my_uid|,
\verb|my_gid| and \verb|my_perm| fields read from the record and the
final parameter specifies which kind of access the user wants to
perform.

\syntax{myro.chkPerm(uid TINYINT UNSIGNED, my\_uid TINYINT UNSIGNED,
  my\_gid TINYINT UNSIGNED, my\_perm TINYINT UNSIGNED, what CHAR(1))}

\begin{parameters}
\param{uid}{TINYINT UNSIGNED}: user ID;
\param{my\_uid}{TINYINT UNSIGNED}: user ID of owner of the record;
\param{my\_gid}{TINYINT UNSIGNED}: group ID of the group to which the
record belongs to;
\param{my\_perm}{TINYINT UNSIGNED}: permission specification;
\param{what}{CHAR(1)}: ``r'' for read access, ``w'' for write access.
\end{parameters}

\return{BOOL}: 1 if the user can access the record, 0 otherwise.

%
\subsection{myro.defgid}
Return the default group ID for the current user.

\syntax{myro.defgid()}

\return{TINYINT UNSIGNED}: default group ID.

%
\subsection{myro.defgrp}
Return the default group name for the current user.

\syntax{myro.defgrp()}

\return{CHAR(50)}: default group name.

%
\subsection{myro.defperm}
Return the default permission specification for the current user.

\syntax{myro.defperm()}

\return{TINYINT UNSIGNED}: default permission specification.



\subsection{myro.fmtPerm}
Return a string representation of a permission specification. A read
permission is identified by character ``r'', a write permission by
character ``w''. The string is made up of 6 characters, the first two
refer to access for the owner of the record, the third and
fourth to access for users belonging to the group, the last two
characters for all other users.
 
\syntax{myro.fmtPerm(perm TINYINT UNSIGNED)}

\begin{parameters}
\param{perm}{TINYINT UNSIGNED}: permission specification.
\end{parameters}

\return{CHAR(6)}: string representation of permission.

%
\subsection{myro.gid2grp}
Return the group name for a given group ID.

\syntax{myro.gid2grp(gid TINYINT UNSIGNED)}

\begin{parameters}
\param{gid}{TINYINT UNSIGNED}
\end{parameters}

\return{CHAR(50)}: group name, or NULL if the \verb|gid| does not exist.




\subsection{myro.grp2gid}
Return the group id for a given group name.

\syntax{myro.grp2gid(grp CHAR(50))}

\begin{parameters}
\param{grp}{CHAR(50)}: group name.
\end{parameters}

\return{TINYINT UNSIGNED}: group ID, or NULL if the group does not
exist.


\subsection{myro.is\_root}
Return a flag telling if current user is ``root''.

\syntax{myro.is\_root()}

\return{BOOL}: 1 if the current user is ``root'', 0 otherwise.

%
\subsection{myro.is\_su}
Returns the ``super user'' flag of a given user ID.

\syntax{myro.is\_su(uid TINYINT UNSIGNED)}

\begin{parameters}
\param{uid}{TINYINT UNSIGNED}: user ID;
\end{parameters}

\return{BOOL}: ``super user'' flag.

%
\subsection{myro.listGroups}
Return a list of groups to which a user belongs to, given its user ID.

\syntax{myro.listGroups(uid TINYINT UNSIGNED)}

\begin{parameters}
\param{uid}{TINYINT UNSIGNED}: user ID.
\end{parameters}

\return{VARCHAR(200)}: a string with a list of group names.

%
\subsection{myro.myuser}
Return the user name of the user who is calling the function.

\syntax{myro.myuser()}

\return{CHAR(50)}: user name of the current user.



\subsection{myro.perm}
Return a permission specification given its string representation.
This function performis the inverse codification of the
\verb|myro.fmtPerm| function.

\syntax{myro.perm(perm CHAR(6))}

\begin{parameters}
\param{perm}{CHAR(6)}: string representation of permission.
\end{parameters}

\return{TINYINT UNSIGNED}: permission specification.

%
\subsection{myro.print\_priv}
Show the list of MySQL grants for a user.

\syntax{CALL myro.print\_priv(usr CHAR(50));}

\begin{parameters}
\param{usr}{CHAR(50)}: user name.
\end{parameters}

%
\subsection{myro.su}
Return the ``super user'' flag for the current user.

\syntax{myro.su()}

\return{BOOL}:``super user'' flag.

%
\subsection{myro.uid}
Return the uid of the user who is calling the function.

\syntax{myro.uid()}

\return{TINYINT UNSIGNED}: user ID of the current user.

%
\subsection{myro.uid2defgid}
Return the default group ID of a given user ID.

\syntax{myro.uid2defgid(uid TINYINT UNSIGNED)}

\begin{parameters}
\param{uid}{TINYINT UNSIGNED}: user ID.
\end{parameters}

\return{TINYINT UNSIGNED}: default group ID.

%
\subsection{myro.uid2defperm}
Return default permission specification of a given user ID.

\syntax{myro.uid2defperm(uid TINYINT UNSIGNED)}

\begin{parameters}
\param{uid}{TINYINT UNSIGNED}: user ID.
\end{parameters}

\return{TINYINT UNSIGNED}: default permission specification.

%
\subsection{myro.uid2usr}
Return the user name of a given user ID.

\syntax{myro.uid2usr(uid TINYINT UNSIGNED)}

\begin{parameters}
\param{uid}{TINYINT UNSIGNED}: user ID.
\end{parameters}

\return{CHAR(50)}: user name, or NULL if the \verb|uid| doesn't exist.

%
\subsection{myro.uid\_member\_of\_anygroup}
Check if a user is member of special group ``anygroup''.

\syntax{myro.uid\_member\_of\_anygroup(uid TINYINT UNSIGNED)}

\begin{parameters}
\param{uid}{TINYINT UNSIGNED}: user ID.
\end{parameters}

\return{BOOL}: 1 if the user is member of ``anygroup'', 0 otherwise.

%
\subsection{myro.uid\_member\_of\_gid}
Check if a user is member of the group identified by \verb|gid|.

\syntax{myro.uid\_member\_of\_gid(uid TINYINT UNSIGNED, gid TINYINT UNSIGNED)}

\begin{parameters}
\param{uid}{TINYINT UNSIGNED}: user ID;
\param{gid}{TINYINT UNSIGNED}: group ID.
\end{parameters}

\return{BOOL}: 1 if the user is member of the group identified by \verb|gid|,
0 otherwise.



\subsection{myro.uid\_member\_of\_grp}
Check if a user is member of a group.

\syntax{myro.uid\_member\_of\_grp(uid TINYINT UNSIGNED, grp CHAR(50))}

\begin{parameters}
\param{uid}{TINYINT UNSIGNED}: user ID.
\param{grp}{CHAR(50)}: group name.
\end{parameters}

\return{BOOL}: 1 if the user is member of the group, 0 otherwise.

%
\subsection{myro.users}
Show a list of all user accounts along with their properties and the
groups they belong to.

\syntax{CALL myro.users();}



\subsection{myro.groups}
Shows a list of all defined groups with their descriptions.

\syntax{CALL myro.groups();}






\subsection{myro.usr2defgid}
Returns the user name of a given user name.

\syntax{myro.usr2defgid(usr CHAR(50))}

\begin{parameters}
\param{usr}{CHAR(50)}: user name.
\end{parameters}

\return{TINYINT UNSIGNED}: default group ID.

%
\subsection{myro.usr2uid}
Return the user ID of a given user name.

\syntax{myro.usr2uid(usr CHAR(50))}

\begin{parameters}
\param{usr}{CHAR(50)}: user name.
\end{parameters}

\return{TINYINT UNSIGNED}: user ID, or NULL if the user doesn't
exist.



\subsection{myro.usr\_descr}
Return the description of a given user name.

\syntax{myro.usr\_descr(usr CHAR(50))}

\begin{parameters}
\param{usr}{CHAR(50)}: user name.
\end{parameters}

\return{CHAR(50)}: description if any has been given, NULL otherwise.

%
\subsection{myro.usr\_email}
Return the e-mail address of a given user name.

\syntax{myro.usr\_email(usr CHAR(50))}

\begin{parameters}
\param{usr}{CHAR(50)}: user name.
\end{parameters}

\return{CHAR(50)}: e-mail address if any has been given, NULL
otherwise.

\end{document}

